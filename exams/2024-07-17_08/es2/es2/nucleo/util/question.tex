Vogliamo permettere ai processi di livello sistema di essere notificati quando si verificano
dei particolari eventi all'interno del modulo sistema. Ci limitiamo a considerare solo gli
eventi corrispondenti alla terminazione (o abort) di un processo di livello utente.
I processi di livello sistema interessati a questo tipo di eventi devono prima {\em registrarsi}
tramite una nuova primitiva \verb|evreg()|;
da quel punto in poi verranno {\em notificati} ogni volta che un
processo utente termina (o abortisce).

Per realizzare il meccanismo modifichiamo la primitiva \verb|wfi()| in modo che possa attendere
notifiche di eventi, oltre che richieste di interruzione. La primitiva modificata deve restituire
un valore con il seguente significato:
\begin{itemize}
 \item 1: \`e stata ricevuta una richiesta di interruzione (comportamento normale della \verb|wfi()|);
 \item 2: \`e stata ricevuta una notifica di terminazione;
 \item 3: entrambe le cose (notifica di terminazione e richiesta di interruzione).
\end{itemize}
Se la \verb|wfi()| restituisce 2 o 3, il processo deve poi {\em rispondere} alla notifica invocando
la primitiva \verb|evget()|, che restituisce l'id del processo terminato. La primitiva \verb|evget()|
pu\`o essere invocata pi\`u volte (anche senza aver prima invocato \verb|wfi()|) e non \`e mai
bloccante: se non ci sono notifiche pendenti si limita a restituire 0.

Il meccanismo, come descritto, impone anche di modificare gli handler, in quanto ora pu\`o accadere
che un processo invochi \verb|wfi()|, inviando l'EOI all'APIC e bloccandosi, e poi si risvegli a causa
di una notifica. Una richiesta di interruzione pu\`o dunque arrivare mentre il processo non \`e
bloccato dentro la \verb|wfi()|; in quel caso l'handler non pu\`o mettere il processo forzatamente in
esecuzione, ma deve limitarsi a settare un flag nel descrittore del processo. Il processo noter\`a
questo flag e agir\`a di conseguenza la prossima volta che invoca \verb|wfi()|.

Pi\`u processi possono registarsi per gli eventi, e ciascuno di essi deve ricevere tutte le notifiche
generate dal momento in cui si \`e registrato in poi. Diciamo che la notifica di un evento \`e {\em in corso} se i
processi registrati sono stati notificati, ma non hanno ancora risposto tutti. 
Quando tutti i processi registrati hanno risposto, diciamo che la notifica \`e {\em completata}.
Un nuovo evento pu\`o essere notificato solo dopo che la notifica del precedente \`e stata completata.
Infine, per evitare che gli id dei processi terminati vengano riusati prima che i processi registrati
abbiano avuto il tempo di riceverli, i processi terminati vengono distrutti solo al completamento
della notifica.

Per realizzare il meccanismo aggiungiumo i seguenti campi al descrittore di processo:
\begin{verbatim}
    bool registrato;
    bool notificato;
    bool bloccato;
    bool ricevuto_intr;
\end{verbatim}
Dove: \verb|registrato| \`e true se il processo \`e registrato per la notifica degli eventi;
\verb|notificato| \`e true se il processo ha ricevuto una notifica a cui non ha ancora risposto;
\verb|bloccato| \`e true se il processo \`e bloccato nella \verb|wfi()|;
\verb|ricevuto_intr| \`e true se \`e arrivata una richiesta di interruzione mentre il processo non
era bloccato nella \verb|wfi()|.

Aggiungiamo inoltre le seguenti variabili globali:
\begin{verbatim}
    des_proc *in_notifica;
    natq risposte_mancanti;
    des_proc *terminati;
\end{verbatim}
Dove: \verb|in_notifica| punta al descrittore del processo (terminato) la cui notifica \`e
ancora in corso (nullptr se non ci sono notifiche in corso);
\verb|risposte_mancanti| conta quanti processi registrati devono ancora rispondere alla
notifica in corso (0 se non ci sono notifiche in corso);\
\verb|terminati| \`e una coda di processi terminati la cui notifica \`e stata rimandata
perch\'e ce n'era gi\`a un'altra in corso.

Modifichiamo gli handler e la \verb|wfi()| come descritto e aggiungiamo le seguenti primitive (invocabili solo da livello sistema):
\begin{itemize}
  \item \verb|bool evreg()|: registra il processo per la ricezione delle notifiche; restituisce
  false in caso di errore (processo gi\`a registrato, notifica in corso);
  \item \verb|natq evget()|: risponde ad eventuali notifiche; restituisce l'id di un processo terminato,
  o 0 se non ci sono notifiche o se il processo non era registrato.
\end{itemize}
Modificare il file \verb|sistema.cpp| per completare le parti mancanti.
