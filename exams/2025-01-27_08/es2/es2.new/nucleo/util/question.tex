Colleghiamo al sistema delle periferiche PCI di tipo \verb|ce|, con vendorID \verb|0xedce| e deviceID \verb|0x1234|.
Ogni periferica \verb|ce| usa 16 byte nello spazio di I/O a partire dall'indirizzo base specificato nel
registro di configurazione BAR0, sia $b$.

Le periferiche \verb|ce| sono semplici periferiche di uscita con un registro TBR (Transmit Buffer Register),
nel quale \`e possibile scrivere il prossimo byte da inviare.
La perifica invia una richiesta di
interruzione quando ha concluso la tramissione del contenuto di TBR\@; fino ad allora, il registro
TBR \`e occupato e ulteriori scritture vengono ignorate.

Nel sistema \`e installata un'unica perififerica \verb|ce|, ma vogliamo fare in modo che
gli utenti possano usarne diverse versioni ``virtuali'', dette VCE, per non dover attendere il completamento
delle trasmissioni.
Ciascuna periferica VCE contiene un buffer (realizzato con un array circolare)
che pu\`o contenere un certo numero di byte
(al massimo \verb|VCE_BUFSIZE|) in attesa di essere trasmessi sulla periferica reale.
Dopo la conclusione di ogni trasmissione, il processo esterno associato alla periferica estrae
un byte da una delle VCE attive e lo trasmette.

Le VCE sono private per processo. I processi che vogliono usare una VCE devono prima allocare la
propria istanza usando la primitiva:
\begin{verbatim}
  bool vcenew()
\end{verbatim}
La primitiva restituisce \verb|false| se non \`e stato possibile allocare la VCE\@.
A quel punto il processo pu\`o usare la primitiva
\begin{verbatim}
 void vcewrite(char c)
\end{verbatim}
per inviare un byte nella VCE\@. La primitiva non attende la conclusione del trasferimento, ma pu\`o
bloccare il processo in attesa che si liberi spazio per il byte nel buffer della VCE\@.
La primitiva \verb|vcenew()| abortisce il processo se questo aveva gi\`a una VCE; la primitiva \verb|vcewrite()|
lo abortisce se non ce l'aveva.

Per descrivere le periferiche \verb|ce| e \verb|vce| aggiungiamo le seguenti strutture dati al modulo I/O:
\begin{verbatim}
struct vce_des {
    char buf[VCE_BUFSIZE];
    natl head;
    natl tail;
    natl n;
    natl sync;
    bool waiting;
    bool terminated;
};
struct ce_des {
    vce_des *vces[MAX_PROC];
    bool busy;
    ioaddr iTBR;
    natl mutex;
} ce;
\end{verbatim}
La struttura \verb|vce_des| descrive i buffer interni
alle \verb|vce|: \verb|buf| \`e l'array circolare;
i byte vanno inseriti all'indice \verb|tail| ed estratti
dall'indice \verb|head|; il campo \verb|n| conta il numero
di byte contenuti nell'array; quando un processo vuole inserire
un byte, ma il buffer \`e pieno, pone \verb|waiting| a \verb|true|\
e si sospende sul semaforo di sincronizzazione \verb|sync|;
il campo \verb|terminated| \`e true se il processo propietario della VCE
\`e terminato.

La struttura \verb|ce_des| descrive la periferica \verb|ce|:
l'array \verb|vces|, indicizzato dai pid dei processi, contiene i puntatori alla \verb|vce_des| di ogni processo
(\verb|nullptr| per i processi che non esistono o non hanno una VCE);
il campo \verb|busy| \`e \verb|true| quando c'\`e una trasmissione in corso; il campo
\verb|iTBR| contiene l'indirizzo del registro TBR;
il campo \verb|mutex| \`e l'indice di un semaforo di mutua esclusione per
l'accesso al \verb|ce_des| e a tutte le VCE\@. Per permettere ai vari processi
di usare le VCE mentre \`e in corso una tramissione precedente, il \verb|mutex|
deve essere occupato solo per il tempo strettamente necessario.

Le VCE dei processi devono essere allocate nello heap del modulo I/O (tramite \verb|new|)
e deallocate (tramite \verb|delete|) quando il loro processo proprietario termina. 
L'allocazione avviene nella \verb|vcenew()|, ma la responsabilit\`a della deallocazione
ricade sul processo esterno associato alla periferica \verb|ce|: il processo esterno deve
deallocare la VCE quando il processo proprietario e terminato e tutti i byte contenuti nella
VCE sono stati tramessi. Per sapere quando il proprietario termina, il processo
esterno sfrutta un meccanismo di notifica che \`e stato aggiunto al modulo sistema.
Al momento dell'attivazione, il processo \`e stato registrato (tramite 
un nuovo parametro della \verb|activate_pe()|) per la ricezione di notifiche di terminazione.
Da quel momento in poi, ogni volta che
un processo utente termina, il modulo sistema notificher\`a il processo esterno, in particolare
rimettendolo in esecuzione se si era bloccato nella \verb|wfi()|.
La prima volta che va in esecuzione, e ogni volta che si sveglia
dalla \verb|wfi()|, il processo esterno deve invocare la primitiva \verb|evget()| per sapere come mai
\`e stato risvegliato. La primitiva restituisce 0 in caso di richiesta di interruzione (caso normale)
e un valore maggiore di zero in caso di notifica di terminazione. Nel secondo caso, il valore restituito
\`e il pid del processo terminato (se si chiama ripetutamente \verb|evget()| senza ripassare dalla \verb|wfi()|,
la primitiva restituisce \verb|0xFFFFFFFF|).

Modificare il \verb|io.cpp| in modo da realizzare le parti mancanti.
